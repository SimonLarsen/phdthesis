In this thesis, we presented five manuscripts related to the analysis of molecular profiling data. The scope of the work in this thesis was quite broad, but all of it was aimed at improving computational analysis of molecular profiling data in one way or another.

In the first manuscript, we introduced CytoMCS, a heuristic algorithm and Cytoscape app for finding the maximum common edge subgraph of two or more networks. The method used an iterated local search algorithm to maximize an edge conservation score that aims to not only maximize the number of fully conserved edges, but also partially conserved edges. At the time of publication, to our knowledge, no other maximum common subgraph methods designed for large networks (\textgreater 1000 nodes), which made it difficult to gauge the quality of the computed solutions.
Recent evaluations have demonstrated that topology-only alignment methods applied to PPI networks produce alignments with relatively little biological relevance compared to methods incorporating sequence information \cite{Gligorijevic2015,Malod-Dognin2017}. With CytoMCS we instead wanted to provide a readily available tool for solving a well-known graph theoretical problem with wide applicability. In future iterations of the method, one could consider incorporating biological similarity information into the fitness function as well. However, the local search implementation utilizes optimizations predicated on edges having unit weight and relaxing these assumptions is likely going to significantly affect performance, in which case a different computational approach may be more appropriate.

In manuscript two we described CoNVaQ, a new tool for performing CNV-based association studies provided as an R package and an online web interface.
CoNVaQ was mainly developed to be used in two cancer studies not included in this thesis.
In \citeauthor{Canto2019} \cite{Canto2019} we applied the statistical model to analyze genomic profiles from locally advanced rectal cancer tumors in order to identify CNV regions associated with response to neoadjuvant 5-fluorouracil-based chemo and radiotherapy.
In \citeauthor{Laufer-Amorim2019} \cite{Laufer-Amorim2019} we instead applied the query-based test to identify recurrent CNVs in a canine prostate cancer cohort.
While these studies demonstrate the utility of CoNVaQ, there are several ways it could be improved in future versions: currently, only discrete labels (gain, loss and LOH) are supported. Future versions could add support for numeric values as well to better utilize available information. In addition to Fisher's exact test, the statistical model could also include a regression-based analysis. This would enable the inclusion of other covariates to adjust for confounding factors.

Manuscript three presented a systemic evaluation of the relationship between gene expression in \emph{E. coli} and the transcriptional regulatory interactions. Using gene expression data from 805 samples under a wide range of conditions, we evaluated the expression of genes coding for transcription factors against targets reported in RegulonDB. We came to the surprising conclusion that both activating and repressing interactions were associated with a modest positive correlation, and that a random network model was not significantly more inconsistent with observed expression levels than the RegulonDB network.
While these findings are significant, further studies are necessary to elucidate the relationship between transcription factor activity and regulatory influence. A similar evaluation could be carried out using time series data in order to discover any time-delayed relationships. Given the limited correlation between mRNA and protein abundance previously reported \cite{Rogers2008,Maier2009,Schwanhaeusser2011,Ghazalpour2011}, measuring transcription factor activity using proteomics technology rather than transcriptomics may also reveal a relationship that is more congruent with the regulatory network.

In manuscript four we introduced Grand Forest, a decision tree ensemble-based method for discovering disease-gene modules in molecular interaction networks using gene expression data. We compared our method to four state of the art methods in five microarray gene expression data sets and found that the genes selected by Grand Forest were consistently more biologically relevant, but further investigation indicated that the results were largely driven by bias in the PPI network. While one could attempt to mitigate this bias by, for instance, adjusting for degree or betweenness centrality, that still would not adequately address the underlying issue. While it seems evident that well-known disease-related proteins have many reported interactors, at least in part, due to research bias, one could also reasonably argue that deregulation of hub proteins is more likely to result in an adverse phenotype as it affects many biological processes. Until we know how much of the \enquote{hubbiness} observed in PPI networks can be attributed to research bias, correcting for node degree could mean we are correcting for an important topological property.
We also observed that the proportion of differentially expressed genes in the reference disease-gene sets were remarkably close to the proportion in non-disease genes, which could point to a potential issue with the evaluation. Differential expression may often be too far downstream to be a reliable indicator of causality. Alternatively, the disease-gene sets obtained from KEGG may not accurately represent the actual disease pathway. Regardless, this issue does not affect other applications of \emph{de novo} network enrichment such as finding biomarkers or drug targets where downstream effects are highly relevant.
Despite these problems, I believe Grand Forest is a promising first step towards decision-tree based module discovery. In addition to being non-parametric, the use of decision trees may help capture interaction effects not identified by existing methods \cite{Lunetta2004,McKinney2006,Wright2016}.
%We also demonstrated how Grand Forest can be used for unsupervised application by applying it to lung adenocarcinoma, clustering subjects based on a 20-gene module.

In chapter six we presented HiGAna, a gene hierarchy-based approach for finding cellular functions and components enriched with genetic variants associated with a phenotype. The method uses a principal component regression-based framework for finding enrichment while correcting for confounding variables and avoiding collinearity due to linkage disequilibrium. We evaluated HiGAna using genotype data from the COPDGene cohort and a gene hierarchy built from Gene Ontology. We discovered several gene sets associated with COPD including two mechanisms not implicated in the standard GWAS results.
We only validated HiGAna against other self-contained tests and not competitive tests as they test different null hypotheses and are thus not comparable.
Self-contained tests have been shown to suffer from inflation under high overall heritability--especially for large gene sets \cite{Leeuw2016}. Future work should explore the possibility of adding a competitive test to HiGAna as well, to enable analysis in instances where a self-contained null hypothesis is not appropriate.
A major issue with using the Gene Ontology as a biological reference is that a large number of genes are only annotated with highly general terms. Variants near such genes generally will not contribute to the analysis since enrichment across highly general terms is unlikely. Recent work has demonstrated that a gene hierarchy can be generated from pairwise gene similarity data (e.g. co-expression and genetic interactions) in \emph{S. cerevisiae} \cite{Dutkowski2012,Kramer2014}. I believe future work should focus on constructing such a hierarchy for human cells to use in enrichment analysis. Such a hierarchy could provide a less biased alternative to Gene Ontology that does not rely on human curation, and may provide more granular annotations for genes that are currently poorly annotated in GO. Computing such a hierarchy from human data is likely going to be challenging, due to the diversity of human cells and significant molecular differences between different tissues, but this could be mitigated by generating tissue-specific hierarchies instead.

In conclusion, in this thesis, we introduced computational methods for finding patterns in molecular profiling data that are associated with a trait. Two of the described methods, HiGAna and Grand Forest, incorporate secondary information in the form of a network or an ontology to obtain a simple model of how proteins are organized in the cell. While this drastically reduces the number of gene sets we have to test for enrichment, it also makes the methods highly vulnerable to any biases present in the network or ontology data. Improving available network and ontology data is thus essential to improving the reliability of methods relying on such data going forward. We also introduced a heuristic algorithm for the maximum common subgraph problem that, most notably, was also able to detect partially conserved edges. Finally, we presented a systemic evaluation of gene expression data in \emph{E. coli} and demonstrated how expression patterns, when viewed as a whole, were not consistent with reported transcriptional regulatory interactions.
Overall, these results demonstrate that bioinformatics tools can aid the discovery of novel biological and biomedical insights, and the integration of biological knowledge bases with experimental data continues to be a promising avenue for finding clinically relevant associations. However, while biological knowledge bases like BioGRID, KEGG and Gene Ontology are constantly growing and improving, they still suffer from both research bias and technical bias. It is thus vital that researchers critically evaluate how well this data represents our understanding of cell biology, and consider how potential biases will be reflected in their results.
