With the advent of high-throughput omics technologies, it is now possible to simultaneously measure thousands of molecules, such as DNA, RNA or proteins, to draw a detailed picture of the cell. Molecular profiling data is, however, complex and noisy and sophisticated methods are necessary to effectively uncover biologically and clinically relevant patterns. This thesis is a collection of separate manuscripts all of which are related to the development of computational methods for analyzing complex biomedical data.  

The first manuscript introduces an iterated local search algorithm for finding the maximum common edge subgraph of two or more large networks. The algorithm optimizes a novel conservation score allowing it to discover both fully and partially conserved edges. The method is provided as an app for the Cytoscape platform.

The next manuscript introduces CoNVaQ, a method for performing copy number variation (CNV)-based genome-wide association studies. The tool provides an algorithm for segmenting CNV calls into discrete regions and two models for finding associations: a statistical significance test using Fisher's exact test and a query-based model. The method is demonstrated by finding variants associated with HPV-status in a penile cancer cohort.

Manuscript three describes a systematic evaluation of the relationship between regulatory interactions and measured gene expression levels in \emph{E. coli}. The study aims to test the assumption that an up- or downregulation of a transcription factor will result in a change in the expression of its targets. We find that both activating and repressing interactions are associated with a very modest positive correlation and that random network models are as consistent with expression data as the real network. We discuss possible reasons for this conclusion.

The fourth Manuscript introduces a decision tree ensemble-based method for integrating a molecular interaction network with gene expression data to find an enriched subnetwork. We evaluate the method against other state of the art \emph{de novo} network enrichment methods and find that it finds dense, more biologically relevant gene modules. However, the results appeared to be mainly driven by bias in the network topology rather than relevant expression patterns.

The final manuscript describes a framework for analyzing genetic variants using a hierarchical model of the cell. We construct a gene hierarchy based on Gene Ontology and search for enriched cellular components or processes using a generalized linear model. We also introduce two conditional tests for filtering terms that are redundant or driven by a single gene. We apply the method to analyze a chronic obstructive pulmonary disease cohort and find two mechanisms not implicated in a standard GWAS analysis. We further discuss the limitations of using Gene Ontology for enrichment analysis with respect to term size and bias.

In conclusion, we developed methods for aiding bioinformatics analysis of biomedical data. Our results demonstrate that while computational analysis of molecular profiling data, especially when integrated with secondary information from network or pathway databases, can effectively help discover novel biological insights, bias and technical artifacts in the underlying data can significantly influence results.
